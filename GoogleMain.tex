\documentclass[12pt]{article}
\usepackage{graphicx}

\usepackage{amsmath}
\usepackage[top=1.4in, bottom=1.4in, left=1.4in, right=1.4in]{geometry}

\usepackage{color}
\usepackage[usenames,dvipsnames,svgnames,table]{xcolor}
\usepackage[colorlinks=true,
            linkcolor=red,
            urlcolor=blue,
            citecolor=gray]{hyperref}

\begin{document}
%\maketitle

\begin{center}
 \LARGE SAMUEL C. TENKA \\
 \normalsize Sophomore at University of Michigan, Ann Arbor (2015-2016) \\
 \href{mailto:samtenka@umich.edu}{samtenka@umich.edu} \\
 \href{http://sctenka.blogspot.com}{sctenka.blogspot.com} \\
 \vspace{0mm}
\end{center}

\begin{section}{Practicum Essays}
\begin{description}
\item[Question 1] %What developed your interest in computer science? How will the EP program support your future goals?
%What developed your interest in computer science? How will the EP program support your future goals?

\emph{What developed your interest in computer science? How will the EP program support your future goals?}

I just met my hero: Douglas Hofstadter. He was visiting Ann Arbor, giving a talk 
celebrating the life and work John Holland, a professor here at Michigan. I was


My interests sprawl. 

On my bookshelf sits a certain book

I started programming in middle school. It began as a way to entertain friends:
my first two memorable programs were a chatbot and a text-adventure game.
Around the same time, I read about the P=NP conjecture and that
the design of algorithms was nontrivial (a difficult and interesting subject).

I want to work on large-scale machine-learning and artificial intelligence projects.
Subgoals for this include advancing my programming skills as far as possible, and
learning .

The Engineering Practicum program is a great way to develop these skills: with all of Google's projects
as large-scale as they are, the program seems an especially good match for my interests.


\item[Question 2] %Tell us about a time you’ve used your strongest coding language. Please go into detail about your experience using this technical language (for example project, competition, website).
%Tell us about a time you’ve used your strongest coding language. Please go into detail about your experience using this technical language (for example project, competition, website).

I like drawing. The problem is, when photographing my drawings,
since my lightbulbs are nonisotropic and at finite distance, I
get a nonuniform background. Luckily, I'm also a native
speaker of C++. Challenge: can we separate content from background---graphite from shadows---in a given bitmap? 

\begin{figure}[h] \label{fig:original}
   \centering
   \includegraphics[width=5cm]{original}
   \caption{A photographed drawing.}
\end{figure}

Pencil in hand, I wrote some constraints: at each pixel, the shadow is
brighter than the shadow+graphite. Its brightness varies, but at a
near-constant gradient. Given those constraints, our guess for the
shadow should be as dark as possible. In mathematical form, these give
the objective function that guided my program's gradient descent.

I used a bitmap library I'd written from scratch for other projects.
A few for-loops later, I was ready to test my program.

It didn't work.

\begin{figure}[h] \label{fig:shadowbad}
   \centering
   \includegraphics[width=5cm]{shadow_bad}
   \caption{Unsuccessfully computed shadows.}
\end{figure}

\pagebreak


Now the fun began: improving on the design, seeing new things. Debugging
methodically, I displayed the paper at each timestep, pinpointed the
coordinates where errors became visible, stepped through execution. . . 
the objective function and gradient descent were implemented correctly,
and I was confident in my bitmap library. Perhaps the objective function
itself was faulty? Indeed, it turned out that in discretized space-time,
my derivatives misbehaved. After some tuning of parameters, I fixed the bug.

\begin{figure}[h]
   \centering
   \includegraphics[width=5cm]{shadow} \includegraphics[width=5cm]{graphite}
   \caption{Final decomposition into shadows (left) and graphite (right).}
\end{figure}


\item[Question 3] %The EP program is committed to increasing diversity within the technology industry. Why do you feel having a diverse workforce is important and what can Google do to further this goal in technology?
%The EP program is committed to increasing diversity within the technology industry. Why do you feel having a diverse workforce is important and what can Google do to further this goal in technology?

\emph{The EP program is committed to increasing diversity within the technology industry. Why do you feel having a diverse workforce is important and what can Google do to further this goal in technology?}


John Holland's schema theorem


\end{description}
\end{section}


\begin{section}{Objectives}
 Apply my programming skills in a summer internship. 
 
 I'm especially interested in the mathematical aspects of algorithm design, for example in machine learning; to such projects I would bring a strong mathematical background of proof and problem-solving.
\end{section}

\begin{section}{Qualifications}
 \subsubsection*{Technical Coursework}
  My studies focus on CS, physics, and pure math. I've taken the following courses, earning a majority of A+'s; courses planned for academic year 2015-16 are in grey:
  \begin{description}
   \item Intro Programming (280), Machine Learning (545), {\color{gray} Artificial Intelligence (492),
         Algo.s and Data Structures (281), OO and Advanced Programming (381), Circuits and Signals (215, 216), Design Project: Automatic Newspaper-Reader (Engr 355).}

  \item Honors Mechanics and E\&M (160, 260 + labs), Statistical Mechanics (510), Modern Physics (390),
        Quantum Mechanics (511, 512), {\color{gray} Quantum Field Theory I (513).}

  \item Honors Math/Analysis (295-396), Differential Equations (316), Probability (425), Math for Finance (423),
        Coding Theory (567), Honors Algebra (493, 494), Explorations in Math (389),
        {\color{gray} Commutative Rings (614,615), Combinatorics (565, 566).}
  \end{description}
  Note: some of the above courses I took while I was in high-school, just for fun. However, I did perform all coursework and was graded on the same scale as official students. Instructor email addresses are available upon request.


 \subsubsection*{Work Experience}
  \begin{description}
  \item Math Tutor at Eastern Michigan University (2011-2014)
  \item Course Assistant for Math 389 (Winters 2015, {\color{gray}2016})
  \item REU Undergrad Researcher in Math (Summer 2015), resulting in \href{http://arxiv.org/abs/1510.08337}{this paper}, written with Stella Gastineau under the mentorship of Andrew Snowden.
  \end{description}

 \subsubsection*{Programming Experience (6 Years)}
  Fluent in Python, C++. Experience in C, C\#, Java, Perl, HTML, PHP, JavaScript.
  
  I've been creating personal projects such as games, text editors, natural language processors, and image processors for several years. 
  
  Combining my passions for physics and computational efficiency, I also have experience in scientific simulations: for example, I've written optimized n-body simulators, Monte-Carlo simulations to study phase transitions, and various numerical PDE solvers. And recently, I've started exploring web development\textemdash my skills here are
  currently limited, but I'm excited to learn more.

  Selected Projects:
  \begin{itemize}
   \item \href{http://sctenka.blogspot.com/2013/08/magnification-art.html}{Image-Magnifier}
   \item \href{http://sctenka.blogspot.com/2014/06/impression.html}{Artistic Filter: Impressionism}
   \item \href{http://sctenka.blogspot.com/2014/07/paper.html}{Automatic Eraser}
   \item \href{http://tweetsense.herokuapp.com/}{TweetSense: A Sentiment Analyzer for MHacks 5}
   \item \href{http://sctenka.blogspot.ch/2015/08/red-ball.html}{3D Printing!}
   \item Work in Progress: Chess Engine based not on Game-Tree Analysis but on Recognition of Machine-Learned Patterns.
  \end{itemize}
\end{section}

\begin{section}{References}
 Available upon request.
\end{section}

\end{document}
