%Tell us about a time you’ve used your strongest coding language. Please go into detail about your experience using this technical language (for example project, competition, website).

I like drawing. The problem is, when photographing my drawings,
since my lightbulbs are nonisotropic and at finite distance, I
get a nonuniform background. Luckily, I'm also a native
speaker of C++. Challenge: can we separate content from background---graphite from shadows---in a given bitmap? 

\begin{figure}[h] \label{fig:original}
   \centering
   \includegraphics[width=5cm]{original}
   \caption{A photographed drawing.}
\end{figure}

Pencil in hand, I wrote some constraints: at each pixel, the shadow is
brighter than the shadow+graphite. Its brightness varies, but at a
near-constant gradient. Given those constraints, our guess for the
shadow should be as dark as possible. In mathematical form, these give
the objective function that guided my program's gradient descent.

I used a bitmap library I'd written from scratch for other projects.
A few for-loops later, I was ready to test my program.

It didn't work.

\begin{figure}[h] \label{fig:shadowbad}
   \centering
   \includegraphics[width=5cm]{shadow_bad}
   \caption{Unsuccessfully computed shadows.}
\end{figure}

\pagebreak


Now the fun began: improving on the design, seeing new things. Debugging
methodically, I displayed the paper at each timestep, pinpointed the
coordinates where errors became visible, stepped through execution. . . 
the objective function and gradient descent were implemented correctly,
and I was confident in my bitmap library. Perhaps the objective function
itself was faulty? Indeed, it turned out that in discretized space-time,
my derivatives misbehaved. After some tuning of parameters, I fixed the bug.

\begin{figure}[h]
   \centering
   \includegraphics[width=5cm]{shadow} \includegraphics[width=5cm]{graphite}
   \caption{Final decomposition into shadows (left) and graphite (right).}
\end{figure}
